% AThesis v2.2

% این قالب بر اساس فرمت پایان‌نامه‌ها و رساله‌های تحصیلات تکمیلی دانشگاه صنعتی اصفهان تهیه شده است.

% ارسال نظرات جهت بهبود قالب:
% a101.fakhari@gmail.com

% توصیه می‌شود که از توزیع تک‌لایو (TexLive) استفاده شود:
% http://tug.org/texlive/acquire-iso.html

% از نصب بودن فونت B ZAR و B Zar Bold بر روی سیستم خود اطمینان حاصل کنید. اگر به مشکل پیدا نکردن فونت برخوردید، از وجود این فونت ها در پوشه Windows/Fonts از طریق CMD مطمین شوید.
% کمپایلر ادیتور را روی XeLaTeX قرار دهید.

% موفق باشید.
% امین فخاری
% ‌دی‌ماه 1394
% -----------------------------------------------------------------------------------

% نکات:

% برای آن‌که پردازش فایل و مشاهده خروجی در هنگام نوشتن پایان‌نامه آسان‌تر و سریع‌تر انجام شود، انجام موارد زیر توصیه می گردد:
% الف) فصل‌ها و بخش‌هایی که در حال نوشتن آن‌ها نیستید را غیر فعال کنید. به‌عنوان مثال، در این قالب، این دستورات را می‌توان در صورت عدم نیاز با اضافه کردن % به طور موقت غیرفعال کرد:
% \MakeTitlePage
% \MakeFarsiSignaturePage
% \clearpage
\thispagestyle{empty}
\newgeometry{left=3cm,right=4cm,top=7cm}

{\BZarScaleOne
{\fontsize{20pt}{0}\selectfont
\noindent
% عنوان تشکر و قدردانی---------------------------------------------------------
تشکر و قدردانی
% ؛---------------------------------------------------------
}}
\vspace{0.5cm}

{\BZarScaleOne
{\fontsize{12pt}{0.9cm}\selectfont % Zar 13
\noindent
% متن تشکر و قدردانی---------------------------------------------------------
خدایا تو را شاکرم به خاطر امروزم که به من عطا فرمودی...






% ؛---------------------------------------------------------
}}

\restoregeometry
% \MakeCopyRightPage
% \clearpage
\thispagestyle{empty}
\newgeometry{left=3cm,right=4cm,top=7cm}

{\BZarScaleOne
{\fontsize{28pt}{0}\selectfont
\noindent
% تقدیم اثر---------------------------------------------------------
تقـدیم به
\\[1cm]
\hspace*{1cm}
همسرم به مهربانی فرشته
% ؛---------------------------------------------------------
}}
		
\restoregeometry
% \MakeTableOfContents
% \MakeListOfFigures
% \MakeListOfTables
% \MakeFarsiAbstract
% \input{Chapters/Chapter#}
% \MakeAppendices
% \input{Chapters/Appendices}
% \MakeEnglishAbstract
% \MakeEnglishSignaturePage
% ب) از گزینه draft برای فراخوانی کلاس استفاده کنید. یعنی
% \documentclass[a4paper,fleqn,10pt,oneside,draft]{book}
% این گزینه حالت چرکنویس را ایفا می‌کند و بر روی بسته‌های مختلف اثرهای متفاوتی دارد. به‌عنوان مثال: به جای شکل، تنها چهارچوب آن نمایش داده شود، لینک‌های hyperref غیر فعال گردد، فایل‌های خارجی را در بسته listings اضافه نمی‌کند و ... و همه این موارد سبب کاهش زمان اجرا و حجم فایل می‌شود.

% در صورتی که میخواهید به سطر بعد بروید اما نمیخواهید بین دو کلمه‌ای که نوشتید فاصله بیفتد کافی است در انتهای خط اول  (بدون فاصله) کاراکتر % را اضافه کنید. با این عمل، لاتک خط فاصله ایجاد شده در اثر تغییر سطر را به عنوان توضیح اضافه یا کامنت در نظر میگیرد و در خروجی اعمال نمی‌کند.

% توصیه می‌شود از شکل‌های برداری با فرمت PDF استفاده شود. این کار علاوه بر افزایش کیفیت رسال/پایان‌نامه/گزارش، باعث کاهش حجم شکل‌ها (و در نتیجه  کاهش حجم فایل نهایی) و همچنین کاهش زمان پردازش می‌شود.

% در این قالب سعی شده است که از تمامی بخش‌های موجود در پایان‌نامه‌ها نمونه‌ای آورده شود.

\documentclass[a4paper,fleqn,10pt,oneside]{book}
%-----------------------------
% بسته‌های دلخواه را در این قسمت اضافه نمایید:

%-----------------------------
\usepackage{Settings/AThesisStyle}
%-----------------------------
% دستورهای مورد نیاز را در این قسمت اضافه نمایید:

%-----------------------------

\begin{document}

\pagestyle{plain}
\pagenumbering{adadi}
\setcounter{page}{2}

% ░░░░░░░▒▒▒▒▒▒▓▓▓▓ In the Name of Allah ▓▓▓▓▒▒▒▒▒▒░░░░░░░
\clearpage
\thispagestyle{empty}
\begin{figure}[t]
\centering
\includegraphics[scale=1.3]{Settings/Allah.pdf}
\end{figure}

% ░░░░░░░▒▒▒▒▒▒▓▓▓▓ Title Page ▓▓▓▓▒▒▒▒▒▒░░░░░░░
\DepartmentFa{دانشکده مهندسی مکانیک}
\ThesisTypeFa{رساله} % Or \ThesisTypeFa{پایان‌نامه} Or \ThesisTypeFa{پیشنهادیه پایان‌نامه}
\DegreeFa{دکتری} % Or \DegreeFa{کارشناسی ارشد}
\FieldFa{مهندسی مکانیک}
\YourFullnameFa{امین فخاری}
\FirstSupervisorFa{دکتر مهدی کشمیری}
\SecondSupervisorFa{دکتر راهنمای دوم} % Optional (Remove It If You Don't Have)
\YearFa{1394}
\TitleFa{
تحلیل و کنترل لغزش در جابجایی اجسام در تماس با سطوح هموار
\\[0.4cm]
توسط انگشتان نرم
}
% اگر عنوان رساله طولانی بود، در دو خط به صورت نشان داده شده تقسیم شود.

\MakeTitlePage

% ░░░░░░░▒▒▒▒▒▒▓▓▓▓ Signature - Farsi ▓▓▓▓▒▒▒▒▒▒░░░░░░░
\Prefix{آقای} %\Prefix{خانم}
\DateFa{1394/10/6}
\FirstAdvisorFa{دکتر مشاور اول}
\SecondAdvisorFa{دکتر مشاور دوم} % Optional (Remove It If You Don't Have)
\FirstExaminerFa{دکتر داور اول}
\SecondExaminerFa{دکتر داور دوم} % Optional (Remove It If You Don't Have)
\ThirdExaminerFa{دکتر داور سوم} % Optional (Remove It If You Don't Have)
\FourthExaminerFa{دکتر داور چهارم} % Optional (Remove It If You Don't Have)
\FifthExaminerFa{دکتر داور پنجم} % Optional (Remove It If You Don't Have)
\DeanOfDepartmentFa{دکتر تحصیلات تکمیلی دانشکده}

\MakeFarsiSignaturePage

% ░░░░░░░▒▒▒▒▒▒▓▓▓▓ Acknowledgments ▓▓▓▓▒▒▒▒▒▒░░░░░░░
\clearpage
\thispagestyle{empty}
\newgeometry{left=3cm,right=4cm,top=7cm}

{\BZarScaleOne
{\fontsize{20pt}{0}\selectfont
\noindent
% عنوان تشکر و قدردانی---------------------------------------------------------
تشکر و قدردانی
% ؛---------------------------------------------------------
}}
\vspace{0.5cm}

{\BZarScaleOne
{\fontsize{12pt}{0.9cm}\selectfont % Zar 13
\noindent
% متن تشکر و قدردانی---------------------------------------------------------
خدایا تو را شاکرم به خاطر امروزم که به من عطا فرمودی...






% ؛---------------------------------------------------------
}}

\restoregeometry

% ░░░░░░░▒▒▒▒▒▒▓▓▓▓ CopyRight ▓▓▓▓▒▒▒▒▒▒░░░░░░░
\MakeCopyRightPage

% ░░░░░░░▒▒▒▒▒▒▓▓▓▓ Dedication ▓▓▓▓▒▒▒▒▒▒░░░░░░░
\clearpage
\thispagestyle{empty}
\newgeometry{left=3cm,right=4cm,top=7cm}

{\BZarScaleOne
{\fontsize{28pt}{0}\selectfont
\noindent
% تقدیم اثر---------------------------------------------------------
تقـدیم به
\\[1cm]
\hspace*{1cm}
همسرم به مهربانی فرشته
% ؛---------------------------------------------------------
}}
		
\restoregeometry

% ░░░░░░░▒▒▒▒▒▒▓▓▓▓ Table of Contents/Figures/Tables ▓▓▓▓▒▒▒▒▒▒░░░░░░░
\MakeTableOfContents
\MakeListOfFigures
\MakeListOfTables

% ----------------------------------------------------------------------------
\clearpage
\pagestyle{myheadings}
\pagenumbering{arabic}
\setcounter{page}{1}

% ░░░░░░░▒▒▒▒▒▒▓▓▓▓ Abstract - Farsi ▓▓▓▓▒▒▒▒▒▒░░░░░░░
\input{Chapters/AbstractFa}
\MakeFarsiAbstract

% ░░░░░░░▒▒▒▒▒▒▓▓▓▓ Chapters ▓▓▓▓▒▒▒▒▒▒░░░░░░░
\clearpage
\baselineskip=0.9cm

\input{Chapters/Chapter1}
\input{Chapters/Chapter2}
\input{Chapters/Chapter3}

% ░░░░░░░▒▒▒▒▒▒▓▓▓▓ Appendices ▓▓▓▓▒▒▒▒▒▒░░░░░░░
\MakeAppendices
\input{Chapters/Appendices}

% ░░░░░░░▒▒▒▒▒▒▓▓▓▓ References ▓▓▓▓▒▒▒▒▒▒░░░░░░░
\MakeReferences
\bibliographystyle{Settings/ModifiedIEEEtranFa}
\bibliography{References}

% ░░░░░░░▒▒▒▒▒▒▓▓▓▓ Abstract - English ▓▓▓▓▒▒▒▒▒▒░░░░░░░
\DepartmentEn{Department of Mechanical Engineering}
\DegreeEn{Doctor of Philosophy (PhD)} % Or \DegreeEn{Master of Science (MSc)} 
\YourFullnameEn{Amin Fakhari}
\YourEmailAddress{a.fakhari@me.iut.ac.ir}
\DateEn{December 27, 2015}
\FirstSupervisorEn{Mehdi Keshmiri, Prof.}
\FirstSupervisorEmailAddress{user1@cc.iut.ac.ir}
\SecondSupervisorEn{Second Supervisor, Prof.} % Optional (Remove It If You Don't Have)
\SecondSupervisorEmailAddress{user2@cc.iut.ac.ir} % Optional (Remove It If You Don't Have)
\TitleEn{Slippage Analysis and Control in Manipulation of Objects \\[0.2cm] in Contact with Even Surfaces Using Soft Fingers}
% اگر عنوان رساله طولانی بود، در دو خط به صورت نشان داده شده تقسیم شود.

\input{Chapters/AbstractEn}
\MakeEnglishAbstract

% ░░░░░░░▒▒▒▒▒▒▓▓▓▓ Signature - English ▓▓▓▓▒▒▒▒▒▒░░░░░░░
\FirstAdvisorEn{First Advisor, Assoc. Prof.}
\SecondAdvisorEn{Second Advisor, Assist. Prof.} % Optional (Remove It If You Don't Have)
\FirstExaminerEn{First Examiner, Prof.}
\SecondExaminerEn{Second Examiner, Prof.} % Optional (Remove It If You Don't Have)
\ThirdExaminerEn{Third Examiner, Prof.} % Optional (Remove It If You Don't Have)
\FourthExaminerEn{Fourth Examiner, Prof.} % Optional (Remove It If You Don't Have)
\FifthExaminerEn{Fifth Examiner, Prof.} % Optional (Remove It If You Don't Have)
\DeanOfDepartmentEn{Dean, Prof.}

\MakeEnglishSignaturePage

\end{document} 
